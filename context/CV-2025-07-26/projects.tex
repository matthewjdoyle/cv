    \section{Research Projects}

        \begin{twocolentry}{
            2020 -- 2024
        }
            \textbf{PhD Research: Simulating Quantum Turbulence}
            \begin{highlights}
                \item Developed and analysed novel simulations (FORTRAN, Python) of quantum vortex dynamics.
                \item Optimised solver algorithms, reducing simulation runtime from months to days for large mesh sizes.
                \item Presented research as (\href{https://qfs2023.org/awards/}{award winning}) posters at numerous UK and international conferences.
                \item Utilised HPC clusters and remote connection to several unused laboratory computers (revived with Linux), connected to a single data storage system.
            \end{highlights}
        \end{twocolentry}

        \vspace{0.2 cm}

        \begin{twocolentry}{
            2022 -- 2024
        }
            \textbf{PhD Research: Visualising Quantum Turbulence}
            \begin{highlights}
                \item Conducted risk assessments, operated rotating cryostat ($<$ 1 K) and collected data from experiments.
                \item Developed video processing pipelines (Python, OpenCV) to detect and track tracer particles.
                \item Performed batch data analysis (Python, pandas) to extract vortex dynamics from observed particle motions.
                \item Collaborated in creation of a machine learning-based (TensorFlow) particle tracking system.
            \end{highlights}
        \end{twocolentry}

        \vspace{0.2 cm}

        \begin{twocolentry}{
            2020 -- 2022
        }
            \textbf{PhD Research: Designing Superfluid $^4$He Flow Experiments}
            \begin{highlights}
                \item Designed (CAD) and collaborated with technicians to construct a novel experimental apparatus.
                \item Built real-time data visualisation and remote experimental control software (Python, LabVIEW).
                \item Performed calibration measurements using classical (normal) fluids.
            \end{highlights}
        \end{twocolentry}


        \vspace{0.2 cm}

        \begin{twocolentry}{
            2019 -- 2020
        }
            \textbf{MSci Thesis: Particle Tracking in a Compact Linear Lepton Collider}
            \begin{highlights}
                \item Evaluated track reconstruction algorithm efficiencies for CERN detector geometry.
                \item Simulated particle collisions using Geant4 on HPC clusters and generated analytics in Python.
                \item Thesis awarded commendation for outstanding research quality.
            \end{highlights}
        \end{twocolentry} 

        \vspace{0.2 cm}

        \begin{twocolentry}{
            2019
        }
        \textbf{Marching Cubes for Fermi-Surface Calculations}
        \begin{highlights}
            \item Implemented a marching cubes algorithm to improve precision for electronic structure calculations by 1,000.
            \item Developed for a FORTRAN codebase, optimised using an OpenMP parallel computing approach.
            \item Visualised the Fermi surfaces of various materials using ggplot and matplotlib scripts.
        \end{highlights}
        \end{twocolentry}


